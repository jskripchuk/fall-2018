%%%%%%%%%%%%%%%%%%%%%%%%%%%%%%%%%%%%%%%%%
% Short Sectioned Assignment
% LaTeX Template
% Version 1.0 (5/5/12)
%
% This template has been downloaded from:
% http://www.LaTeXTemplates.com
%
% Original author:
% Frits Wenneker (http://www.howtotex.com)
%
% License:
% CC BY-NC-SA 3.0 (http://creativecommons.org/licenses/by-nc-sa/3.0/)
%
%%%%%%%%%%%%%%%%%%%%%%%%%%%%%%%%%%%%%%%%%

%----------------------------------------------------------------------------------------
%	PACKAGES AND OTHER DOCUMENT CONFIGURATIONS
%----------------------------------------------------------------------------------------

\documentclass[paper=a4, fontsize=11pt]{scrartcl} % A4 paper and 11pt font size

\usepackage[T1]{fontenc} % Use 8-bit encoding that has 256 glyphs
\usepackage{fourier} % Use the Adobe Utopia font for the document - comment this line to return to the LaTeX default
\usepackage[english]{babel} % English language/hyphenation
\usepackage{amsmath,amsfonts,amsthm} % Math packages

\usepackage{lipsum} % Used for inserting dummy 'Lorem ipsum' text into the template

\usepackage{sectsty} % Allows customizing section commands
\allsectionsfont{\centering \normalfont\scshape} % Make all sections centered, the default font and small caps

\usepackage{fancyhdr} % Custom headers and footers
\pagestyle{fancyplain} % Makes all pages in the document conform to the custom headers and footers
\fancyhead{} % No page header - if you want one, create it in the same way as the footers below
\fancyfoot[L]{} % Empty left footer
\fancyfoot[C]{} % Empty center footer
\fancyfoot[R]{\thepage} % Page numbering for right footer
\renewcommand{\headrulewidth}{0pt} % Remove header underlines
\renewcommand{\footrulewidth}{0pt} % Remove footer underlines
\setlength{\headheight}{13.6pt} % Customize the height of the header

\numberwithin{equation}{section} % Number equations within sections (i.e. 1.1, 1.2, 2.1, 2.2 instead of 1, 2, 3, 4)
\numberwithin{figure}{section} % Number figures within sections (i.e. 1.1, 1.2, 2.1, 2.2 instead of 1, 2, 3, 4)
\numberwithin{table}{section} % Number tables within sections (i.e. 1.1, 1.2, 2.1, 2.2 instead of 1, 2, 3, 4)

\setlength\parindent{0pt} % Removes all indentation from paragraphs - comment this line for an assignment with lots of text

%----------------------------------------------------------------------------------------
%	TITLE SECTION
%----------------------------------------------------------------------------------------

\newcommand{\horrule}[1]{\rule{\linewidth}{#1}} % Create horizontal rule command with 1 argument of height

\title{	
\normalfont \normalsize 
\huge Homework 1 \\ % The assignment title
\horrule{0.4pt} \\[0.5cm] % Thick bottom horizontal rule
}

\author{James Skripchuk} % Your name

\date{\normalsize\today} % Today's date or a custom date

\begin{document}

\maketitle % Print the title

%----------------------------------------------------------------------------------------
%	PROBLEM 1
%----------------------------------------------------------------------------------------

\section{Problem 1}

Prove by mathematical induction that 
\begin{equation} \label{eq:1}
\sum_{i=1}^{p} i^3 = \frac{1}{4}p^2(p+1)^2
\end{equation}

We will be using induction on $p$
%\lipsum[2] % Dummy texx



%------------------------------------------------

\subsection{Base Case}

Proving holds for $p=1$

\begin{align}
\sum_{i=1}^{1} i^3 = 1
\end{align}
\begin{align}
\frac{1}{4}(1^2)(1+1)^2 = \frac{1}{4}(2)^2 = \frac{4}{4} = 1
\end{align}

\subsection{Induction Hypothesis}
We will be using weak induction. Assume that \ref{eq:1} holds for $p=n$, thus
\begin{equation} \label{eq:2}
\sum_{i=1}^{n} i^3 = \frac{1}{4}n^2(n+1)^2
\end{equation} 

\subsection{Induction Step}
%Left side
\begin{align} 
\sum_{i=1}^{n+1} i^3 	&= \sum_{i=1}^{n} i^3 + (n+1)^3 \tag*{}\\
&=\frac{1}{4}n^2(n+1)^2 + (n+1)^3 \tag*{By Induction Hypothesis}\\
&=(n+1)^2(\frac{1}{4}n^2+n+1) \tag*{}
\end{align}

Which is the same as showing that

\begin{align} 
\begin{split}
\sum_{i=1}^{n+1} i^3 	&= \frac{1}{4}(n+1)^2(n+2)^2\\
&= (n+1)^2\frac{1}{4}(n^2+4n+4) \\
&=(n+1)^2(\frac{1}{4}n^2+n+1))\\
\end{split}	
\end{align}
\subsection{Conclusion}

Since \ref{eq:1} holds true for the base case, and \ref{eq:1} is also true for $p=n$, it follows that \eqref{eq:1} is true for all $p\geq1$
%------------------------------------------------

%----------------------------------------------------------------------------------------
%	PROBLEM 2
%----------------------------------------------------------------------------------------
\section{Problem 2}
Consider the sequence of real numbers defined by the relations

\begin{equation}
R_1 = 1 
\end{equation}
\begin{equation}
R_x = \sqrt[]{1 + 2R_{x-1}} \tag*{for all $x > 1$}
\end{equation}

Prove by mathematical induction that for $R_x<4$ for all $x\geq1$ 
\subsection{Base Case}
\begin{equation}
R_1 = 1 < 4
\end{equation}
Thus the base case holds and we can continue using weak induction
\subsection{Induction Hypothesis}
Assume that $R_x < 4$ holds for $x = n$, therefore
\begin{equation}
R_n < 4
\end{equation}
\subsection{Induction Step}
\begin{align} 
R_{n+1} 	&= \sqrt[]{1 + 2R_{n}} \tag*{}\\
			&< \sqrt[]{1 + 2(4)} \tag*{By Induction Hypothesis} \\
			&< \sqrt[]{9}\tag*{} \\
			&< 3\tag*{} \\
\end{align}

Thus

\begin{equation}
R_{n+1} < 3 < 4
\end{equation}

\subsection{Conclusion}

Since $R_x < 4$ holds for the base case of $x = 1$ and we have proven that $R_n < 4$ for all $x=n$, it follows that $R_x < 4$ is true for all $x>1$

%----------------------------------------------------------------------------------------
%	PROBLEM 3
%----------------------------------------------------------------------------------------
\section{Problem 3}
Consider the following definition of a non-empty binary tree:

\begin{verbatim}
;; a Non-Empty Binary Tree [NEBT] is either
;; - a leaf l, with no internal structure or children, or
;; - (make-node left[NEBT] right[NEBT]) an internal node with two children, 
;; each of which is a NEBT.
\end{verbatim}
Now consider this NEBT as a graph: each leaf or internal node is a VERTEX in V, and every
internal node has two unique EDGES in E linking the internal node to two other NEBTs. Prove that 
\begin{equation}\label{eq:3}
\mid{V}\mid{} = \mid{E}\mid{} + 1
\end{equation}
\subsection{Base Case}
A leaf $l$ has no internal structure or children, so thus it has one vertex and no edges. Therefore $1=0+1$. The base case is true so we can continue.
\subsection{Induction Hypothesis}
Let us take three NEBTs, denoted $n,l,r$. Assume that \ref{eq:3} holds for these three NEBTs, therefore

\begin{equation}
\mid{V_n}\mid{} = \mid{E_n}\mid{} + 1
\end{equation}
\begin{equation}
\mid{V_l}\mid{} = \mid{E_l}\mid{} + 1
\end{equation}
\begin{equation}
\mid{V_r}\mid{} = \mid{E_r}\mid{} + 1
\end{equation}
\subsection{Induction Step}
Let us take NEBT $n$ and create a new NEBT by calling (make-node $r$ $l$) on an external node of $n$. This new NEBT created shall be denoted $m$. We would like to show that 

\begin{equation}
\mid{V_m}\mid{} = \mid{E_m}\mid{} + 1
\end{equation}

We know that $m$ will hold the following properties due to the nature of binary trees
\begin{equation}
\mid{V_m}\mid{} = \mid{V_n}\mid{} + \mid{V_r}\mid{} + \mid{V_l}\mid{}
\end{equation}

\begin{equation}\label{eq:4}
\mid{E_m}\mid{} = \mid{E_n}\mid{} + \mid{E_r}\mid{} + \mid{E_l}\mid{}+2
\end{equation}

We can now continue with the induction

\begin{align} 
\mid{V_m}\mid{} &= \mid{V_n}\mid{} + \mid{V_r}\mid{} + \mid{V_l}\mid{}\\
				&= (\mid{E_n}\mid{} + 1) + (\mid{E_r}\mid{} + 1) + (\mid{E_l}\mid{} + 1)\tag*{By Induction Hypothesis}\\
				&= (\mid{E_n}\mid{} + \mid{E_r}\mid{} + \mid{E_l}\mid{} +2) +1 \tag*{}\\
				&= \mid{E_m}\mid{}+1 \tag*{By \ref{eq:4}} \\
\end{align}
\subsection{Conclusion}

Since \ref{eq:3} holds for the base case, and \ref{eq:3} also holds for any arbitrary structure, we've proven we can construct any other structure where \ref{eq:3} holds.

%----------------------------------------------------------------------------------------
%	PROBLEM 4
%----------------------------------------------------------------------------------------
\section{Problem 4}
\subsection{Base Case}
\subsection{Induction Hypothesis}
\subsection{Induction Step}
\subsection{Conclusion}



%----------------------------------------------------------------------------------------

\end{document}