%%%%%%%%%%%%%%%%%%%%%%%%%%%%%%%%%%%%%%%%%
% Short Sectioned Assignment
% LaTeX Template
% Version 1.0 (5/5/12)
%
% This template has been downloaded from:
% http://www.LaTeXTemplates.com
%
% Original author:
% Frits Wenneker (http://www.howtotex.com)
%
% License:
% CC BY-NC-SA 3.0 (http://creativecommons.org/licenses/by-nc-sa/3.0/)
%
%%%%%%%%%%%%%%%%%%%%%%%%%%%%%%%%%%%%%%%%%

%----------------------------------------------------------------------------------------
%	PACKAGES AND OTHER DOCUMENT CONFIGURATIONS
%----------------------------------------------------------------------------------------

\documentclass[paper=a4, fontsize=11pt]{scrartcl} % A4 paper and 11pt font size

\usepackage[T1]{fontenc} % Use 8-bit encoding that has 256 glyphs
\usepackage{fourier} % Use the Adobe Utopia font for the document - comment this line to return to the LaTeX default
\usepackage[english]{babel} % English language/hyphenation
\usepackage{amsmath,amsfonts,amsthm} % Math packages

\usepackage{lipsum} % Used for inserting dummy 'Lorem ipsum' text into the template

\usepackage{sectsty} % Allows customizing section commands
\allsectionsfont{\centering \normalfont\scshape} % Make all sections centered, the default font and small caps

\usepackage{fancyhdr} % Custom headers and footers
\pagestyle{fancyplain} % Makes all pages in the document conform to the custom headers and footers
\fancyhead{} % No page header - if you want one, create it in the same way as the footers below
\fancyfoot[L]{} % Empty left footer
\fancyfoot[C]{} % Empty center footer
\fancyfoot[R]{\thepage} % Page numbering for right footer
\renewcommand{\headrulewidth}{0pt} % Remove header underlines
\renewcommand{\footrulewidth}{0pt} % Remove footer underlines
\setlength{\headheight}{13.6pt} % Customize the height of the header

\numberwithin{equation}{section} % Number equations within sections (i.e. 1.1, 1.2, 2.1, 2.2 instead of 1, 2, 3, 4)
\numberwithin{figure}{section} % Number figures within sections (i.e. 1.1, 1.2, 2.1, 2.2 instead of 1, 2, 3, 4)
\numberwithin{table}{section} % Number tables within sections (i.e. 1.1, 1.2, 2.1, 2.2 instead of 1, 2, 3, 4)

\setlength\parindent{0pt} % Removes all indentation from paragraphs - comment this line for an assignment with lots of text

%----------------------------------------------------------------------------------------
%	TITLE SECTION
%----------------------------------------------------------------------------------------

\newcommand{\horrule}[1]{\rule{\linewidth}{#1}} % Create horizontal rule command with 1 argument of height

\title{	
\normalfont \normalsize 
\huge Homework 1 \\ % The assignment title
\horrule{0.4pt} \\[0.5cm] % Thick bottom horizontal rule
}

\author{James Skripchuk} % Your name

\date{\normalsize\today} % Today's date or a custom date

\begin{document}

\maketitle % Print the title

%----------------------------------------------------------------------------------------
%	PROBLEM 1
%----------------------------------------------------------------------------------------

\section{Problem 1}

Prove by mathematical induction that 
\begin{equation} \label{eq:1}
\sum_{i=1}^{p} i^3 = \frac{1}{4}p^2(p+1)^2
\end{equation}

We will be using induction on $p$
%\lipsum[2] % Dummy texx



%------------------------------------------------

\subsection{Base Case}

Proving holds for $p=1$

\begin{align}
\sum_{i=1}^{1} i^3 = 1
\end{align}
\begin{align}
\frac{1}{4}(1^2)(1+1)^2 = \frac{1}{4}(2)^2 = \frac{4}{4} = 1
\end{align}

\subsection{Induction Hypothesis}
We will be using weak induction. Assume that \ref{eq:1} holds for $p=n$, thus
\begin{equation} \label{eq:2}
\sum_{i=1}^{n} i^3 = \frac{1}{4}n^2(n+1)^2
\end{equation} 

\subsection{Induction Step}
%Left side
\begin{align} 
\sum_{i=1}^{n+1} i^3 	&= \sum_{i=1}^{n} i^3 + (n+1)^3 \tag*{}\\
&=\frac{1}{4}n^2(n+1)^2 + (n+1)^3 \tag*{By Induction Hypothesis}\\
&=(n+1)^2(\frac{1}{4}n^2+n+1) \tag*{}
\end{align}

Which is the same as showing that

\begin{align} 
\begin{split}
\sum_{i=1}^{n+1} i^3 	&= \frac{1}{4}(n+1)^2(n+2)^2\\
&= (n+1)^2\frac{1}{4}(n^2+4n+4) \\
&=(n+1)^2(\frac{1}{4}n^2+n+1))\\
\end{split}	
\end{align}
\subsection{Conclusion}

Since we proved that blash bablah 
%------------------------------------------------

\subsubsection{Heading on level 3 (subsubsection)}

\lipsum[3] % Dummy text

\paragraph{Heading on level 4 (paragraph)}

\lipsum[6] % Dummy text

%----------------------------------------------------------------------------------------
%	PROBLEM 2
%----------------------------------------------------------------------------------------

\section{Lists}

%------------------------------------------------

\subsection{Example of list (3*itemize)}
\begin{itemize}
	\item First item in a list 
		\begin{itemize}
		\item First item in a list 
			\begin{itemize}
			\item First item in a list 
			\item Second item in a list 
			\end{itemize}
		\item Second item in a list 
		\end{itemize}
	\item Second item in a list 
\end{itemize}

%------------------------------------------------

\subsection{Example of list (enumerate)}
\begin{enumerate}
\item First item in a list 
\item Second item in a list 
\item Third item in a list
\end{enumerate}

%----------------------------------------------------------------------------------------

\end{document}